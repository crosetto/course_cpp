\documentclass[aspectratio=43]{beamer}
% Theme works only with a 4:3 aspect ratio
\usetheme{CSCS}

% define footer text
\newcommand{\footlinetext}{P. Crosetto}

% Select the image for the title page
\newcommand{\picturetitle}{cscs_images/image3.pdf}
%\newcommand{\picturetitle}{cscs_images/image5.pdf}
%\newcommand{\picturetitle}{cscs_images/image6.pdf}


% Please use the predifined colors:
% cscsred, cscsgrey, cscsgreen, cscsblue, cscsbrown, cscspurple, cscsyellow, cscsblack, cscswhite

\author{Paolo Crosetto, C2SM/CSCS}
\title{C++11 constexpr keyword}
\subtitle{Advanced C++ for HPC}
\date{\today}

\begin{document}

% TITLE SLIDE
\cscstitle

% EMPTY SLIDE
%% \begin{frame}
%% \end{frame}

% TABLE OF CONTENT SLIDE
% All options for table of contents:
% currentsection, currentsubsection, firstsection=xx, hideallsubsections, hideothersubsections, part=xx, pausesections, pausesubsections, sections=xx, sections={xx-yy}, sections={xx,yy}
%\cscstableofcontents[hideallsubsections]{Title}
\cscstableofcontents{C++11 constexpr keyword}

\begin{frame}[fragile]{Material}
  Clone the github repo
\begin{verbatim}
git clone https://github.com/crosetto/examples_cpp.git
\end{verbatim}
  Most of the code snippets shown are avaliable in
\begin{verbatim}
examples_cpp/Code/Constexpr/
\end{verbatim}
  Exercises are in
\begin{verbatim}
examples_cpp/Exercises/Constexpr/
\end{verbatim}
\end{frame}

\section{Motivations}


\begin{frame}[fragile]{Motivation}
  We want to compute the sum of the first N numbers, $\left(\sum_{i=0}^N i\right)$ both at runtime and at compile time:
  \begin{Cpplisting}[: template metafunction]{}
int sum2(int id){ // regular recursion
  return id>1 ? id+sum(id-1) : id;
};

template <int I>
struct sum{ // template recursion
  static const long int value=I+sum<I-1>::value;
};

template <>
struct sum<1>{
  static const long int value=1;
};

void test(int runtime){
  static_assert(sum<16>::value > 0, "error");
  std::cout<<sum2(runtime)<<"\n";
}
\end{Cpplisting}
\end{frame}

\begin{frame}[fragile]{Motivation}
  We want to compute the sum of the first N numbers, $\left(\sum_{i=0}^N i\right)$ both at runtime and at compile time:
  \begin{Cpplisting}[: template metafunction]{}
constexpr int sum(int id){ // regular recursion
  return id>1 ? id+sum(id-1) : id;
};











void test(int runtime){
  static_assert(sum(16) > 0, "error");
  std::cout<<sum(runtime)<<"Error!\n";
}
\end{Cpplisting}

\end{frame}

\begin{frame}[fragile]{Motivation II}
  \begin{itemize}
  \item ``complicated'' types at compile time
  \item reduced compilation time
  \item see \verb1example_00.cpp1
  \item paste this in your browser:
    \lstset{
      basicstyle=\ttfamily,
      columns=fullflexible,
      keepspaces=true,
      }
    \begin{lstlisting}[breaklines]
https://gcc.godbolt.org/#compilers:!((compiler:g62,options:'-std%3Dc%2B%2B11+-O3++-fconstexpr-depth%3D1000',source:'constexpr+%0A++int+sum2(int+id)%7B%0A++++return+id%3E1+%3F+id%2Bsum2(id-1)+:+id%3B%0A%7D%3B%0A%0Aint+main()%7B%0A++//constexpr+int+ret%3Dsum2(100)%3B%0A++//return+ret%3B%0A++return+sum2(100)%3B%0A%7D%0A')),filterAsm:(commentOnly:!t,directives:!t,labels:!t),version:3
    \end{lstlisting}
  \end{itemize}
\end{frame}

%% \begin{frame}[fragile]\frametitle{Motivations}
%%   We saw lot of template metaprogramming
%%   \begin{itemize}
%%   \item \verb1constexpr1 allows sharing the same code for compile-time and run-time computations;
%%   \item it also simplifies the syntax;
%%   \item it simplifies the compiler error messages (no huge nested types);
%%   \item it can reduce the compilation time.
%%   \end{itemize}
%%   You already knew that
%% \end{frame}

%% \begin{frame}[fragile]\frametitle{Motivations II}
%%   We will focus in this section:
%%   \begin{itemize}
%%   \item on details, corner cases, best practices for the usage of \verb1constexpr1;
%%   \item on reusable patterns, and few involved examples (using other C++11/14 features like variadic templates, integer sequence).
%%   \end{itemize}
%% \end{frame}

\section{The syntax}
\subsection{Details}
\begin{frame}[fragile]\frametitle{Syntax}
  \begin{definition}[constant expression]
  expression that is evaluated at compile-time
  \end{definition}

  Meanings of the \verb1constexpr1 keyword:
  \begin{itemize}
  \item on function return type: the fuction \alert{may} or \alert{MAY NOT} return a constant expression
  \item on variable instance: the instance is a constant expression
  \item when a constexpr function is used in a constant expression, it must return a constant expression
  \end{itemize}
\begin{Cpplisting}[: example\_01.cpp]{}
  constexpr float d=13.2;
  constexpr double f(){return 13.2;}
  static_assert(d==13.2, "");
  static_assert(f()==13.2, "");
\end{Cpplisting}
\end{frame}

\begin{frame}[fragile]\frametitle{Syntax}

  Constraints for constexpr functions:
  \begin{itemize}
  \item no local variables (until C++14)
  \item no if statements (until C++14)
  \item single return statement (until C++14)
  \item only literal arguments
  \item no static variables
  \item cannot be virtual
  \item see cppreference for details
  \end{itemize}
\begin{Cpplisting}[: example]{}
constexpr double f(){ return 13.2;}
\end{Cpplisting}

You can \alert{instantiate your own class} as a constant expression

\end{frame}


\begin{frame}[fragile]\frametitle{Syntax}
  In order to instantiate your own class as a constant expression:
  \begin{itemize}
  \item must have a constexpr constructor;
  \item constexpr constructor must be a constexpr function;
  \item base objects and data members must be constexpr constructable;
  \end{itemize}
NOTE: function arguments cannot be labeled constexpr (it is \alert{not a type qualifier}).
The compiler treats them as constant expressions when:
\begin{itemize}
\item the function is used in a constant expression (e.g. to initialize it)
\item no guarantee otherwise, up to the compiler
\end{itemize}
NOTE: (non-static) data members cannot be labeled constexpr.
They are constant expressions when their object is.
\end{frame}

\begin{frame}[fragile]\frametitle{Syntax}
  \begin{onlyenv}<1-2>
\begin{Cpplisting}[: example\_02.cpp]{}
constexpr int f1(std::vector<int> const& v_){
    return v_.size();
}






int main(){
    std::vector<int> t(1,2,3,4);
    std::cout<<f1(t)<<"\n";
}
\end{Cpplisting}
  \end{onlyenv}
  \only<1-2>{\uncover<2>{Vectors do not have constexpr constructor (allocated on the heap)}}
  \begin{onlyenv}<3-4>
\begin{Cpplisting}[: example\_02.cpp]{}
template<typename ... T>
constexpr auto f2(std::tuple<T...> const& t_){
    return std::get<2>(t_);
}





int main(){
    auto t=std::make_tuple(1,2,3,4);
    std::cout<<f2(t)<<"\n";
}
\end{Cpplisting}
  \end{onlyenv}
  \only<3-4>{\uncover<4>{Tuples have constexpr constructors, std::get is a constexpr function}}
  \begin{onlyenv}<5-6>
\begin{Cpplisting}[: example\_02.cpp]{}
template<typename ... T>
constexpr auto f3(std::tuple<T...> const& t_){
    bool cond=std::get<2>(t_)<std::get<1>(t_);
    if(cond)
        return std::get<2>(t_);
    else
        return std::get<1>(t_);
}

int main(){
    auto t=std::make_tuple(1,2,3,4);
    std::cout<<f3(t)<<"\n";
}
\end{Cpplisting}
  \end{onlyenv}
\only<5-6>{\uncover<6>{Since C++14 can have ifs, local variables, multiple returns}}
\end{frame}

\begin{frame}[fragile]\frametitle{Syntax}
  Define your own literal type:
  \begin{Cpplisting}[: example\_03.cpp]{}
struct my_const_struct{
    template<typename T>
    constexpr my_const_struct(T t_) : m_i(0){
        bool cond=std::get<2>(t_)<std::get<1>(t_);
        if(cond)
            m_i = std::get<2>(t_);
        else
            m_i = std::get<1>(t_);
    }
    constexpr int get() const {return m_i;}
    int m_i;
};

int main(){
    constexpr auto t_=std::make_tuple(1,2,3,4);
    constexpr my_const_struct my_struct(t_);
    std::cout<<my_struct.get()<<"\n";
    static_assert(my_struct.get()==2, "error");
}
  \end{Cpplisting}
\end{frame}


\begin{frame}[fragile]\frametitle{STL}
  Examples of non constexpr:
  \begin{itemize}
  \item \verb1std::vector1
  \item \verb1std::string1
  \item \verb1std::function1
  \item \verb1std::algorithms/iterators1
  \item initializer list (until C++14)
  \item C++11 lambdas (until C++17)
  \end{itemize}
  Examples of constexpr:
  \begin{itemize}
  \item \verb1std::pair1
  \item \verb1std::tuple1
  \item \verb1std::array1
  \item \verb1std::bitset1
  \item See (e.g.) SPROUT library
  \end{itemize}

\end{frame}

\begin{frame}[fragile]
  \begin{itemize}\frametitle{C++17 Extension}
  \item constexpr lambdas
\end{itemize}

\begin{Cpplisting}[: constexpr lambdas]{}
constexpr auto f_(){return 5;}
int main(){
    constexpr auto f = []()->int{ return 5;};
    static_assert(f_()==5, "error"); // this works
    static_assert(f()==5, "error"); // this does not work
}
\end{Cpplisting}
\alert{Until C++17 constexpr lambdas won't be available}

\end{frame}


\subsection{Limits}

\begin{frame}[fragile]\frametitle{Template Limit}
  \begin{itemize}
    \item Template instantiation depth limit (default compiler-dependent)
  \end{itemize}

%% \onslide<1>{
\begin{Cpplisting}[: example\_00.cpp]{}
template <int I>
struct sum{
    static const long int value=I+sum<I-1>::value;
};

template <>
struct sum<1>{
    static const long int value=1;
};

int main(){
    static_assert(sum<901>::value > 0, "error");
}
\end{Cpplisting}
%% }
\end{frame}


\begin{frame}[fragile]\frametitle{Constexpr Limit}
  \begin{itemize}
    \item Compile-time recursion depth limit (default compiler-dependent)
  \end{itemize}

%% \onslide<2>{
\begin{Cpplisting}[: example\_00.cpp]{}






constexpr int sum(int id){
    return id>1 ? id+sum(id-1) : id;
};

int main(){
    static_assert(sum(512) > 0, "error");
}
\end{Cpplisting}
%% }
\end{frame}

\begin{comment}
\begin{frame}[fragile]\frametitle{Example}
\begin{Cpplisting}[: template recursion]{}
constexpr int sum(int id){
    return id>1 ? id+sum(id-1) : id;
};
\end{Cpplisting}
Which of the following would work (and why)?

\begin{Cpplisting}[: template recursion]{}
int main( int argc, char** argv){
  static_assert(sum(16) > 0, "error");
  int arg=atoi(argv[1]);
  if(sum(arg) < 0) std::cout<<"Error!\n";
  static_assert(sum(arg) > 0, "error");
}
\end{Cpplisting}
How would you write this as a template metafunction?
\end{frame}
\end{comment}

\begin{frame}[fragile]\frametitle{Corner Case}
\begin{Cpplisting}[: example\_04.cpp]{}
constexpr int divide(int id) {
    return id>0 ? id/divide(id-1) : id;
};
int main(){
    static_assert(divide(4)>0, "error");
}
\end{Cpplisting}
\begin{onlyenv}<2->
gives the error:
\verb1error: non-constant condition for static assertion1
\end{onlyenv}

\uncover<3>{
\begin{itemize}
  \item Expression with undefined result or throwing exceptions are NOT constant expressions
\end{itemize}

\alert{Obscure error, be careful when checking out-of-bounds indices}
}
\end{frame}


%% \begin{frame}[fragile]
%%   \begin{itemize}
%%     \item initializer_list
%%     \item std string
%%     \item compile time recursion depth limit
%%     \item expression whose result is not defined (may not be detected)
%%     \item lambdas (since c++17)
%%     \item constness (from C++14)
%%   \end{itemize}
%% \end{frame}

\subsection{Corner Case}

\begin{frame}[fragile]\frametitle{Check if Constant Expression}
  trick to determine if a function returns a constant expression\uncover<2>{:
  \alert{use the noexcept} keyword}
\begin{onlyenv}<1>
  \begin{Cpplisting}[: example\_05.cpp]{}
struct test{
  constexpr test(int arg_) : m_val{arg_}{}
  constexpr int get() const {return m_val;}
  int m_val;
};
int main(){
  constexpr test t1_(5);
  test t2_(10);



}
  \end{Cpplisting}
\end{onlyenv}
\begin{onlyenv}<2>
    \begin{Cpplisting}[: example\_05.cpp]{}
struct test{
  constexpr test(int arg_) : m_val{arg_}{}
  constexpr int get() const {return m_val;}
  int m_val;
};
int main(){
  constexpr test t1_(5);
  test t2_(10);
  constexpr bool check_t1_=(noexcept(t1_.get()));
  constexpr bool check_t2_=(noexcept(t2_.get()));
  static_assert(check_t1_==true, "error");
  static_assert(check_t2_==false, "error");}
  \end{Cpplisting}
\end{onlyenv}
\uncover<2>{
  noexcept returns true for constant expressions. Works only when get
  is not declared as noexcept. Otherwise:
  \begin{itemize}
  \item use static asserts to issue error when not constexpr,
  \item SFINAE if you are desperate (and you'll become more desperate).
  \end{itemize}
  }
\end{frame}

\subsection{Integer Sequence}

\begin{frame}[fragile]\frametitle{std::integer\_sequence}
  We know how to transform a variadic pack into a tuple. But how do we get a variadic pack \alert{from} a tuple?

\begin{Cpplisting}[: Pack to Tuple]{}
auto tuple_=std::make_tuple(1., 'b', 'c', 2u, "ciao");
\end{Cpplisting}
I want to create a vector with the size of each element of the tuple, \verb~{8,1,1,4,8}~,
i.e. a way to automatically obtain:
{\tiny\verb~{sizeof(std::get<0>(myTuple)),sizeof(std::get<1>(myTuple)),sizeof(std::get<2>(myTuple)),...}~}

\begin{onlyenv}<1>
\begin{Cpplisting}[: tuple\_to\_pack.cpp]{}







.
\end{Cpplisting}
\end{onlyenv}

\begin{onlyenv}<2>
\begin{Cpplisting}[: tuple\_to\_pack.cpp]{}
template<typename Tuple, int ... Ids>
std::array<int, sizeof...(Ids)>
create_vector(Tuple& tuple_,
              std::integer_sequence<int, Ids ...>){
  return {sizeof(std::get<Ids>(tuple_)) ...};
}

.
\end{Cpplisting}
\end{onlyenv}

\begin{onlyenv}<3>
\begin{Cpplisting}[: tuple\_to\_pack.cpp]{}
template<typename Tuple, int ... Ids>
std::array<int, sizeof...(Ids)>
create_vector(Tuple& tuple_,
              std::integer_sequence<int, Ids ...>){
  return {sizeof(std::get<Ids>(tuple_)) ...};
}
//in the main
create_vector(tuple_, std::make_integer_sequence<int, 5>());
\end{Cpplisting}
\end{onlyenv}
\end{frame}


\section{Reusable Design Examples}
\subsection{Expression Templates / Automatic Differentiation}

\begin{frame}[fragile]\frametitle{Example I: ET/AD}
  Goal: define a simple language for expressing algebraic expressions
  \begin{itemize}
    \item
    \begin{itemize}
    \item define a \emph{ring} algebraic structure, i.e. 2 operations, $*$ and $+$ with usual properties;
    \item be able to lazily evaluate the expression.
    \item \alert{Definition} happens \alert{at compile time};
    \item \alert{Evaluation} happens \alert{either at run time or at compile time}, depending on the argument.
    \item be able to define/evaluate arbitrary combinations of the operators $*$, $+$;
    \end{itemize}
    \item
    \begin{itemize}
    \item define a differentiation rule $D$ with usual properties, and introduce it to the grammar;
    \end{itemize}
  \end{itemize}
  We choose a \emph{polynomial-like} notation for the expression, i.e.
  $f(x)=x(x-x^2+\frac{\partial(x^3+x)}{\partial x})$
  translates to
  \verb1auto f_x=x*(x-x*x+D(x*x*x+x));1.

\end{frame}

\begin{frame}[fragile]\frametitle{Example I: ET/AD}
  We define a placeholder, i.e. a type representing our independent variable $x$.
\begin{Cpplisting}[: p.cpp]{}
struct p{
  constexpr p(){};

  template<typename T>
  constexpr T operator() (T t_) const {
    return t_;
  }
};

constexpr p x=p();
\end{Cpplisting}
Global variable x=p(): to lighten the notation.
\end{frame}


\begin{frame}[fragile]\frametitle{Example I: ET/AD}
  define the plus expression and overload the $+$ operator
\begin{Cpplisting}[: plus.cpp]{}
template <typename T1, typename T2>
struct expr_plus{
  template<typename T>
  constexpr T operator() (T t_) const{
    return T1()(t_)+T2()(t_);
  }
};

template<typename T1, typename T2>
constexpr expr_plus<T1, T2>
operator + (T1 arg1, T2 arg2){
  return expr_plus<T1, T2 >();}
\end{Cpplisting}
NOTE: in real code use protections and namespaces
\end{frame}

\begin{frame}[fragile]\frametitle{Example I: ET/AD}
  define the times expression and overload the $*$ operator
\begin{Cpplisting}[: times.cpp]{}
template <typename T1, typename T2>
struct expr_times{
  template<typename T>
  constexpr T operator() (T t_) const{
    return T1()(t_)*T2()(t_);
  }
};

template<typename T1, typename T2>
constexpr expr_times<T1, T2>
operator * (T1 arg1, T2 arg2){
  return expr_times<T1, T2 >();}
\end{Cpplisting}
\end{frame}

\begin{frame}[fragile]\frametitle{Example I: ET/AD}
  main function:
  \begin{Cpplisting}[: main]{}
int main(){
  constexpr auto expr = x*x*x; // define an expression
  std::cout<< expr(3)<<"\n"; // evaluate it in x=3
}
  \end{Cpplisting}
  $\approx30$ lines of code?
  That was easy.
\end{frame}

\begin{frame}[fragile]\frametitle{Example I: ET/AD}
  Now we want to compute the derivative: we use the \alert{chain rule}
  $$
  \frac{\partial f(g(x))}{\partial x}=\frac{\partial f(g)}{\partial g}\frac{\partial g(x)}{\partial x}
  $$
  i.e. we multiply the derivatives of each expression.

  We define the operator \verb1D1, and the derivation for a generic object T
  \begin{Cpplisting}[: D.cpp]{}
template <typename TT>
struct expr_derivative{
    using value_t = int;

    template <typename T>
    constexpr T operator()(T t_) const {
        return 0;
    }
};

template<typename T>
constexpr expr_derivative<T> D(T) { return expr_derivative<T>();}
  \end{Cpplisting}
\end{frame}

\begin{frame}[fragile]\frametitle{Example I: ET/AD}
  Let's define the derivative of each expression, starting from $\frac{\partial x}{\partial x}=1$

  \begin{Cpplisting}[: Dp.cpp]{}
template<>
struct expr_derivative<p>{

  template <typename T>
  constexpr T operator() (T t_) const{
    return (T) 1;
  }
};
  \end{Cpplisting}
\end{frame}

\begin{frame}[fragile]\frametitle{Example I: ET/AD}
  Derivative of the sum: $\frac{\partial (a+b)}{\partial x}=\frac{\partial (a)}{\partial x}+\frac{\partial (b)}{\partial x}$
  \begin{Cpplisting}[: Dplus.cpp]{}
template <typename T1, typename T2>
struct expr_derivative<expr_plus<T1, T2> >{
    using value_t = decltype(D(T1())+ D(T2()));

    template <typename T>
    constexpr T operator() (T t_) const{
        return value_t()(t_);
    }
};
  \end{Cpplisting}
\end{frame}

\begin{frame}[fragile]\frametitle{Example I: ET/AD}
  Derivative of the product: $\frac{\partial (a*b)}{\partial x}=\frac{\partial (a)}{\partial x}*b+a*\frac{\partial (b)}{\partial x}$
  \begin{Cpplisting}[: Dtimes.cpp]{}
template <typename T1, typename T2>
struct expr_derivative<expr_times<T1, T2> >{

    using value_t = decltype(T1() * D(T2())
                             +
                             D(T1()) * T2());

    template <typename T>
    constexpr T operator()(T t_)  const{
        return value_t()(t_);
    }
};
  \end{Cpplisting}
\end{frame}


\begin{frame}[fragile]\frametitle{Example I: ET/AD}
  Derivative of the derivative: $\frac{\partial^2 a}{\partial x} = \frac{\partial (\frac{\partial a}{\partial x})}{\partial x}$
  \begin{Cpplisting}[: DD.cpp]{}
template <typename T1>
struct expr_derivative<expr_derivative<T1>>{

    using value_t = expr_derivative<typename expr_derivative<T1>::value_t>;

    template <typename T>
    constexpr T operator() (T t_) const{
        return value_t()(t_);
    }
};
  \end{Cpplisting}

  We also define the operator \verb1D1, returning the derivative expression
  \begin{Cpplisting}[: D.cpp]{}
template <typename T1>
constexpr expr_derivative<T1>
D (T1 arg1){
  return expr_derivative<T1>();}
  \end{Cpplisting}
\end{frame}

\begin{frame}[fragile]\frametitle{Example I: ET/AD}
  Now we can extend our main function with the derivative computation:
  \begin{Cpplisting}[: main.cpp]{}
int main(){
  constexpr auto expr = x*x*x; // define an expression
  std::cout<< expr(3)<<"\n"; // evaluate it in x=3
  std::cout<< D(expr)(2)<<"\n"; // evaluate it in x=2
  std::cout<< D(D(expr))(5)<<"\n"; // evaluate it in x=5
}
  \end{Cpplisting}
\end{frame}


\begin{frame}[fragile]\frametitle{Exercise I}
  Exercise: count (at compile time) the number of sums and multiplications of an expression,
  in order for the following to work:

  \begin{Cpplisting}[: Exercise I]{}
std::cout<< "sums: "<< D(expr).sum_ops()<<"\n";
std::cout<< "multiplications: "<< D(expr).mult_ops()<<"\n";
  \end{Cpplisting}
\end{frame}

\subsection{Multidimensional Array Indexing}

\begin{frame}[fragile]\frametitle{Example II: Storage Info}
  Goal:
  \begin{itemize}
  \item Implement a multidimensional array indexing
  \item with arbitrary dimensionality;
  \item with strides computation happening at compile time when all dimension are compile-time constants;
  \item with an index function, returning the index given the dimensions;
  \end{itemize}

  \begin{onlyenv}<1>
    \begin{Cpplisting}[: storage\_main.cpp]{}
int main(){









}
    \end{Cpplisting}
  \end{onlyenv}
  \begin{onlyenv}<2>
  \begin{Cpplisting}[: storage\_main.cpp]{}
int main(){
  //run-time instance
  storage_info<3> meta_(5,4,3);
  std::cout<<"index(1,0,0): "<<meta_.index(1,0,0)<<"\n";//1
  std::cout<<"index(0,1,0): "<<meta_.index(0,1,0)<<"\n";//5
  std::cout<<"index(0,0,1): "<<meta_.index(0,0,1)<<"\n";//20
  //compile-time instance
  constexpr storage_info<2> c_meta_{5,8};
  static_assert(c_meta_.index(1,0)==1, "error");
  static_assert(c_meta_.index(0,1)==5, "error");
}
    \end{Cpplisting}
  \end{onlyenv}
\end{frame}


\begin{frame}[fragile]\frametitle{Example II: Storage Info}
  Formula to compute the global index for dimension n:
  $$
  g_n = index(coord_n, coord_{n-1}, ...., coord_0) =
  $$
  $$
  = coord_n + (dims_{n})(coord_{n-1}) + (dims_{n}dims_{n-1})(coord_{n-2}) + ...
  $$
  which is, for generic d dimensions
  $$
  g_n = coord_n + \sum_{k=1}^{n-1}\left(\prod_{i=0}^{k-1}dims_{n-i}\right)(coord_{n-k}) = coord_n + dims_ng_{n-1}
  $$
  Think recursively:
  $$
  g_k =
  \left\{
  \begin{array}{ll}
      g_{k-1} dims_{k} + coord_{k} & \forall k>0;\\
      coord_0 & k=0
  \end{array}
  \right.
  $$


\end{frame}

\begin{frame}[fragile]\frametitle{Example II: Storage Info}
  This example is written using the C++14 standard, and in particular shows a practical use case for
  variadic templates and \verb1std::integer_sequence1. It relies on recursion (this should be familiar).

  \alert{Reminder (variadic templates):}
  to specify etherogeneous lists of types:
  \begin{Cpplisting}[: variadic templates]{}
  template<typename ... Ints>
  constexpr int index(Ints ... idx){
  \end{Cpplisting}

  \alert{Reminder (integer sequence):}
  to assign an integer index to each of the elements in the \verb1dims_1 ("zipping"):
  \begin{Cpplisting}[: integer sequence]{}
  template<int ... Indices, typename ... Ints>
  constexpr int idx_impl(std::integer_sequence< Indices ... > , Ints ... dims_) const{
    do_something(make_pair(dims_, Indices) ...);
  }
  \end{Cpplisting}
\end{frame}


\begin{frame}[fragile]\frametitle{Example II: Storage Info}
  \begin{Cpplisting}[: storage\_main.cpp]{}
template <int Size>
struct storage_info{

  template <typename ... Dims>
  constexpr storage_info(Dims ... dims_)
    : m_dims{dims_ ...}{}

  static constexpr int size(){return Size;}

  template<int Coord>
  constexpr int dim() const {return m_dims[Coord];}

private:
  int m_dims[Size];
};
  \end{Cpplisting}
\uncover<2>{  We need to implement the \emph{index} method }
\end{frame}

\begin{frame}[fragile]\frametitle{Example II: Storage Info}

  The index member function satisfies the requested API, and calls an helper function \verb~idx_impl~, passing on a new parameter pack: \alert{$n$}

  \begin{Cpplisting}[: index.cpp]{}
template<typename ... Ints>
constexpr int index(Ints ... coords_) const{
  return idx_impl(
    std::make_integer_sequence<int, sizeof...(Ints)>()
    , coords_ ... );
}
  \end{Cpplisting}
  \uncover<2>{  We pass on 2 parameter packs: the \emph{coordinates} $coord$ and the \emph{indices} $k$,
    we have to implement the recursive formula
  $$
    \left\{
    \begin{array}{l}
      g_k = g_{k-1} dims_{k} + coord_{k};\\
      g_0 = coord_0
    \end{array}
    \right.
  $$

  }
\end{frame}

\begin{frame}[fragile]\frametitle{Example II: Storage Info}
  $$
      g_k = g_{k-1} dims_{k} + coord_{k};
  $$
  \begin{Cpplisting}[: strides.cpp]{}
template<int CurK, int ... RestK
         , typename CurC, typename ... RestC>
constexpr int idx_impl(
        std::integer_sequence< int,  CurK, RestK ... >
        , CurC first_coord_, RestC ... coords_) const {

  return first_coord_ // coord_k
  + m_dims[CurK] //dims_k
  * idx_impl(std::integer_sequence<int, RestK ...>()
             , coords_ ...); // recursion
}
  \end{Cpplisting}
\end{frame}


\begin{frame}[fragile]\frametitle{Example II: Storage Info}
$$
  g_0 = coord_0
$$
  \begin{Cpplisting}[: stop\_rec.cpp]{}
template<int Z, typename Coord>
constexpr int idx_impl(std::integer_sequence< int, Z > , Coord coord_) const{
  return coord_;
}
  \end{Cpplisting}
\end{frame}

%% \begin{frame}[fragile]\frametitle{Example II: Strides Computation}
%%   Recursive constexpr member function to compute the strides
%%   \begin{Cpplisting}[: strides computation ]{}
%% template<int First, int ... Indices, typename First_Int, typename ... Ints>
%% constexpr int idx_impl(std::integer_sequence< int,  First, Indices ... > , First_Int first_, Ints ... indices) const {
%%   return (sizeof ... (Indices) > 0)
%%   ? first_ + m_dims[First] * idx_impl(std::integer_sequence<int, Indices ...>(), indices ...)
%%   : first_;
%% }
%%   \end{Cpplisting}
%% \end{frame}

\begin{frame}[fragile]\frametitle{Example II: Main}
  \begin{Cpplisting}[: main ]{}
#include "storage.hpp"
#include <iostream>

int main(){
    constexpr storage_info<5> c_{2,3,4,5,6};
    static_assert(c_.index(1,0,0,0,0)==1, "error");
    static_assert(c_.index(0,1,0,0,0)==2, "error");
    static_assert(c_.index(0,0,1,0,0)==3*2, "error");
    static_assert(c_.index(0,0,0,1,0)==4*3*2, "error");
    static_assert(c_.index(0,0,0,0,1)==5*4*3*2, "error");
}
  \end{Cpplisting}
\end{frame}

\begin{frame}[fragile]\frametitle{Exercise II}
  Abstract the memory layout of the storage, specifying it with the type \verb~layout_map<0,1,3,2,...>~,
  so that the following can work

  \begin{Cpplisting}[: Exercise II]{}
constexpr storage_info<layout_map<4,3,2,1,0> > c_{2,3,4,5,6};

static_assert(c_.index(1,0,0,0,0)==6*5*4*3, "error");
static_assert(c_.index(0,1,0,0,0)==6*5*4, "error");
static_assert(c_.index(0,0,1,0,0)==6*5, "error");
static_assert(c_.index(0,0,0,1,0)==6, "error");
static_assert(c_.index(0,0,0,0,1)==1, "error");
  \end{Cpplisting}
  To do so follow the instructions in the \verb1storage_info.hpp1 source file.
\end{frame}


% CHAPTER SLIDE
% \cscschapter{Chapter Title}

%% % Block style example
%% \begin{frame}{Quick Styles}
%% \begin{columns}
%%     \begin{column}{0.12\paperwidth}
%%     \begin{black0block}{Abc}
%%     \end{black0block}
%%     \begin{black1block}{Abc}
%%     \end{black1block}
%%     \begin{black2block}{Abc}
%%     \end{black2block}
%%     \end{column}

%%     \begin{column}{0.12\paperwidth}
%%     \begin{green0block}{Abc}
%%     \end{green0block}
%%     \begin{green1block}{Abc}
%%     \end{green1block}
%%     \begin{green2block}{Abc}
%%     \end{green2block}
%%     \end{column}

%%     \begin{column}{0.12\paperwidth}
%%     \begin{blue0block}{Abc}
%%     \end{blue0block}
%%     \begin{blue1block}{Abc}
%%     \end{blue1block}
%%     \begin{blue2block}{Abc}
%%     \end{blue2block}
%%     \end{column}

%%     \begin{column}{0.12\paperwidth}
%%     \begin{brown0block}{Abc}
%%     \end{brown0block}
%%     \begin{brown1block}{Abc}
%%     \end{brown1block}
%%     \begin{brown2block}{Abc}
%%     \end{brown2block}
%%     \end{column}

%%     \begin{column}{0.12\paperwidth}
%%     \begin{purple0block}{Abc}
%%     \end{purple0block}
%%     \begin{purple1block}{Abc}
%%     \end{purple1block}
%%     \begin{purple2block}{Abc}
%%     \end{purple2block}
%%     \end{column}

%%     \begin{column}{0.12\paperwidth}
%%     \begin{yellow0block}{Abc}
%%     \end{yellow0block}
%%     \begin{yellow1block}{Abc}
%%     \end{yellow1block}
%%     \begin{yellow2block}{Abc}
%%     \end{yellow2block}
%%     \end{column}

%%     \begin{column}{0.12\paperwidth}
%%     \begin{red0block}{Abc}
%%     \end{red0block}
%%     \begin{red1block}{Abc}
%%     \end{red1block}
%%     \begin{red2block}{Abc}
%%     \end{red2block}
%%     \end{column}

%% \end{columns}
%% \end{frame}

%% % Example
%% \begin{frame}{Formula}
%%     My formula is $E=m\times c^2$
%%     \begin{blue2block}{Are examples in a blue block?}
%%     \begin{itemize}
%%         \item My example
%%         \begin{itemize}
%%             \item for a formula
%%             \begin{itemize}
%%                 \item rocks!
%%             \end{itemize}
%%         \end{itemize}
%%     \end{itemize}
%%     \end{blue2block}

%%     \begin{red2block}{Alert an error in the formula!}
%%     \begin{enumerate}
%%         \item First enumerated item
%%         \begin{enumerate}
%%             \item First first enumerated item
%%             \begin{enumerate}
%%                 \item First first first enumerated item
%%             \end{enumerate}
%%         \end{enumerate}
%%         \item just joking...
%%     \end{enumerate}
%%     \end{red2block}
%% \end{frame}

%% \begin{frame}[fragile]{Code listing}
%% \begin{Cpplisting}[: My code title with some openmp]{}
%% int main (int argc, char *argv[]) {
%%     int nthreads, tid;
%%     #pragma omp parallel private(nthreads, tid) {
%%         tid = omp_get_thread_num();
%%         printf("Hello World from thread = %d\n", tid);
%%         if (tid == 0) {
%%             nthreads = omp_get_num_threads();
%%             printf("Number of threads = %d\n", nthreads);
%%         }
%%     }  /* All threads join master thread and disband */
%%     return 0;
%% }
%% \end{Cpplisting}
%% \begin{Fortranlisting}[: My Fortran code here]{}
%% program variable
%% real :: x ! This defines a variable called x
%% x = 23.6*log(4.2)/(3.0+2.1) ! This gives it a value
%% print *, 'The value of x is ', x
%% end program
%% \end{Fortranlisting}
%% \footnotesize
%% Here some cpp code in the text: \lstinlineCpp{int main(int args, char *argv[])}\\
%% Here some Fortran code in the text: \lstinlineFortran{INTEGER :: Y(1,5)}
%% \end{frame}

% THANK YOU SLIDE

\end{document}
