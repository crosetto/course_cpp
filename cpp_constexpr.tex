
\documentclass[aspectratio=43]{beamer}
% Theme works only with a 4:3 aspect ratio
\usetheme{CSCS}

% define footer text
\newcommand{\footlinetext}{Insert\_Footer}

% Select the image for the title page
\newcommand{\picturetitle}{cscs_images/image3.pdf}
%\newcommand{\picturetitle}{cscs_images/image5.pdf}
%\newcommand{\picturetitle}{cscs_images/image6.pdf}


% Please use the predifined colors:
% cscsred, cscsgrey, cscsgreen, cscsblue, cscsbrown, cscspurple, cscsyellow, cscsblack, cscswhite

\author{Paolo Crosetto, C2SM/CSCS}
\title{The wonders of the constexpr keyword}
\subtitle{Advanced C++ for HPC}
\date{\today}

\begin{document}

% TITLE SLIDE
\cscstitle

% EMPTY SLIDE
%% \begin{frame}
%% \end{frame}

% TABLE OF CONTENT SLIDE
% All options for table of contents:
% currentsection, currentsubsection, firstsection=xx, hideallsubsections, hideothersubsections, part=xx, pausesections, pausesubsections, sections=xx, sections={xx-yy}, sections={xx,yy}
%\cscstableofcontents[hideallsubsections]{Title}
\cscstableofcontents{The wonders of the constexpr keyword}

\section{Motivations}


\begin{frame}[fragile]{Motivation}
  We want to compute the sum of the first N numbers, $\left(\sum_{i=0}^N i\right)$ both at runtime and at compile time:
  \begin{Cpplisting}[: template metafunction]{}
int sum(int id){ // regular recursion
    return id>1 ? id+sum(id-1) : id;
};

template <int I>
struct sum{ // template recursion
    static const long int value=I+sum<I-1>::value;
};

template <>
struct sum<1>{
    static const long int value=1;
};

int main(){
    static_assert(sum<16>::value > 0, "error");
    if(sum(argv[1]) < 0) std::cout<<"Error!\n";
}
\end{Cpplisting}
\end{frame}

  \begin{frame}[fragile]{Motivation}
  We want to compute the sum of the first N numbers, $\left(\sum_{i=0}^N i\right)$ both at runtime and at compile time:
  \begin{Cpplisting}[: template metafunction]{}
constexpr int sum(int id){ // regular recursion
    return id>1 ? id+sum(id-1) : id;
};











int main(){
    static_assert(sum(16) > 0, "error");
    if(sum(argv[1]) < 0) std::cout<<"Error!\n";
}
\end{Cpplisting}

\end{frame}

\subsection{Limits}

\begin{frame}[fragile]\frametitle{Template Limit}
  \begin{itemize}
    \item Template instantiation depth limit (default compiler-dependent)
  \end{itemize}

%% \onslide<1>{
\begin{Cpplisting}[: template recursion]{}
template <int I>
struct sum{
    static const long int value=I+sum<I-1>::value;
};

template <>
struct sum<1>{
    static const long int value=1;
};

int main(){
    static_assert(sum<901>::value > 0, "error");
}
\end{Cpplisting}
%% }
\end{frame}


\begin{frame}[fragile]\frametitle{Constexpr Limit}
  \begin{itemize}
    \item Compile-time recursion depth limit (default compiler-dependent)
  \end{itemize}

%% \onslide<2>{
\begin{Cpplisting}[: constexpr recursion]{}






constexpr int sum(int id){
    return id>1 ? id+sum(id-1) : id;
};

int main(){
    static_assert(sum(512) > 0, "error");
}
\end{Cpplisting}
%% }
\end{frame}

%% \begin{frame}[fragile]\frametitle{Motivations}
%%   We saw lot of template metaprogramming
%%   \begin{itemize}
%%   \item \verb1constexpr1 allows sharing the same code for compile-time and run-time computations;
%%   \item it also simplifies the syntax;
%%   \item it simplifies the compiler error messages (no huge nested types);
%%   \item it can reduce the compilation time.
%%   \end{itemize}
%%   You already knew that
%% \end{frame}

%% \begin{frame}[fragile]\frametitle{Motivations II}
%%   We will focus in this section:
%%   \begin{itemize}
%%   \item on details, corner cases, best practices for the usage of \verb1constexpr1;
%%   \item on reusable patterns, and few involved examples (using other C++11/14 features like variadic templates, integer sequence).
%%   \end{itemize}
%% \end{frame}

\section{The syntax}
\subsection{Details}
\begin{frame}[fragile]\frametitle{Syntax}
  \begin{definition}[constant expression]
  expression that is evaluated at compile-time
  \end{definition}

  Meanings of the \verb1constexpr1 keyword:
  \begin{itemize}
  \item on function return type: the fuction may or MAY NOT return a constant expression;
  \item on variable instance: the instance is a constant expression
  \end{itemize}
\begin{Cpplisting}[: example]{}
  constexpr float d=13.2;
  constexpr double f(){return 13.2;}
  static_assert(d==13.2, "");
  static_assert(f()==13.2, "");
\end{Cpplisting}
\end{frame}

\begin{frame}[fragile]\frametitle{Syntax}

  Constraints for constexpr functions:
  \begin{itemize}
  \item no local variables (until C++14)
  \item no if statements (until C++14)
  \item single return statement (until C++14)
  \item no non-constexpr arguments
  \item no static variables
  \item cannot be virtual
  \item see cppreference for details
  \end{itemize}
\begin{Cpplisting}[: example]{}
constexpr double f(){
    return 13.2;
}
\end{Cpplisting}
\end{frame}

\begin{frame}[fragile]\frametitle{Syntax}
  \begin{onlyenv}<1-2>
\begin{Cpplisting}[: example]{}
constexpr int f1(std::vector<double> const& v_){
    return v_.size();
}






int main(){
    std_vector<int> t(1,2,3,4);
    std::cout<<f1(t)<<"\n";
}
\end{Cpplisting}
  \end{onlyenv}
  \only<1-2>{\uncover<2>{Vectors do not have constexpr constructor (allocated on the heap)}}
  \begin{onlyenv}<3-4>
\begin{Cpplisting}[: example]{}
template<typename ... T>
constexpr auto f2(std::tuple<T...> const& t_){
    return std::get<2>(t_);
}





int main(){
    auto t=std::make_tuple(1,2,3,4);
    std::cout<<f2(t)<<"\n";
}
\end{Cpplisting}
  \end{onlyenv}
  \only<3-4>{\uncover<4>{Tuples have constexpr constructors, std::get is a constexpr function}}
  \begin{onlyenv}<5-6>
\begin{Cpplisting}[: example]{}
template<typename ... T>
constexpr auto f3(std::tuple<T...> const& t_){
    bool cond=std::get<2>(t_)<std::get<1>(t_);
    if(cond)
        return std::get<2>(t_);
    else
        return std::get<1>(t_);
}

int main(){
    auto t=std::make_tuple(1,2,3,4);
    std::cout<<f3(t)<<"\n";
}
\end{Cpplisting}
  \end{onlyenv}
\only<5-6>{\uncover<6>{Since C++14 can have ifs, local variables, multiple returns}}
\end{frame}

\begin{frame}[fragile]\frametitle{Syntax}
  In order to instantiate your own class as a constant expression:
  \begin{itemize}
  \item must contain/inherit from objects with a constexpr constructor;
  \item must have a constexpr constructor;
  \item constexpr constructor must be a constexpr function (see cppreference for details)
  \end{itemize}
NOTE: function arguments cannot be labeled constexpr.
  The compiler treats them as constant expressions when possible.
\end{frame}

\begin{frame}[fragile]\frametitle{Syntax}
  \begin{Cpplisting}[: example]{}
struct my_const_struct{
    template<typename T>
    constexpr my_const_struct(T t_) : m_i(0){
        bool cond=std::get<2>(t_)<std::get<1>(t_);
        if(cond)
            m_i = std::get<2>(t_);
        else
            m_i = std::get<1>(t_);
    }
    constexpr int get() const {return m_i;}
    int m_i;
};

int main(){
    constexpr auto t_=std::make_tuple(1,2,3,4);
    constexpr my_const_struct my_struct(t_);
    std::cout<<my_struct.get()<<"\n";
    static_assert(my_struct.get()==2, "error");
}
  \end{Cpplisting}
\end{frame}

\begin{frame}[fragile]\frametitle{STL}
  Examples of non constexpr:
  \begin{itemize}
  \item \verb1std::vector1
  \item \verb1std::string1
  \item \verb1std::algorithms/iterators1
  \item C++11 lambdas (until C++17)
  \end{itemize}
  Examples of constexpr:
  \begin{itemize}
  \item \verb1std::pair1
  \item \verb1std::tuple1
  \item \verb1std::array1
  \item See (e.g.) SPROUT library
  \end{itemize}

\end{frame}

\begin{comment}
\begin{frame}[fragile]\frametitle{Example}
\begin{Cpplisting}[: template recursion]{}
constexpr int sum(int id){
    return id>1 ? id+sum(id-1) : id;
};
\end{Cpplisting}
Which of the following would work (and why)?

\begin{Cpplisting}[: template recursion]{}
int main( int argc, char** argv){
  static_assert(sum(16) > 0, "error");
  int arg=atoi(argv[1]);
  if(sum(arg) < 0) std::cout<<"Error!\n";
  static_assert(sum(arg) > 0, "error");
}
\end{Cpplisting}
How would you write this as a template metafunction?
\end{frame}
\end{comment}

\begin{frame}[fragile]\frametitle{Corner Cases}
  \begin{itemize}
  \item Expression with undefined result or throwing exceptions are NOT constant expressions
\end{itemize}

\begin{Cpplisting}[: divide by zero]{}
constexpr int divide(int id) {
    return id>-1 ? id/divide(id-1) : id;
};
int main(){
    static_assert(divide(4)>0, "error"); //error
}
\end{Cpplisting}
gives the error:
\verb1error: non-constant condition for static assertion1

\alert{Obscure error, be careful when checking out-of-bounds indices}
\end{frame}

\begin{frame}[fragile]
  \begin{itemize}\frametitle{C++17 Extension}
  \item constexpr lambdas
\end{itemize}

\begin{Cpplisting}[: divide by zero]{}
constexpr auto f_(){return 5;}
int main(){
    constexpr auto f = []()->int{ return 5;};
    static_assert(f_()==5, "error"); // this works
    static_assert(f()==5, "error"); // this does not work
}
\end{Cpplisting}
\alert{Until C++17 constexpr lambdas won't be available}

\end{frame}


%% \begin{frame}[fragile]
%%   \begin{itemize}
%%     \item initializer_list
%%     \item std string
%%     \item compile time recursion depth limit
%%     \item expression whose result is not defined (may not be detected)
%%     \item lambdas (since c++17)
%%     \item constness (from C++14)
%%   \end{itemize}
%% \end{frame}

\subsection{Best Practices}

\begin{frame}[fragile]\frametitle{Check if Constant Expression}
  trick to determine if a function returns a constant expression:
  \alert{use the noexcept} keyword
  \begin{Cpplisting}[: check if constexpr]{}
struct test{
    int m_val;
    constexpr test(int arg_) : m_val{arg_}{}
    constexpr int get() const {return m_val;}
};
int main(){
    constexpr test t1_(5);
    test t2_(10);
    constexpr bool check_t1_=(noexcept(t1_.get()));
    constexpr bool check_t2_=(noexcept(t2_.get()));
    static_assert(check_t1_==true, "error");
    static_assert(check_t2_==false, "error");}
  \end{Cpplisting}
  \verb1noexcept1 returns true for constant expressions. Works only when \verb1get1
  is not declared as noexcept. Otherwise:
  \begin{itemize}
  \item use static asserts to issue error when not constexpr,
  \item SFINAE if you are desperate (and you'll become more desperate).
  \end{itemize}
\end{frame}

\section{Reusable Design Patterns}
\subsection{Expression Templates / Automatic Differentiation}

\begin{frame}[fragile]\frametitle{Example I: ET/AD}
  Goal:
  \begin{itemize}
  \item define a \emph{ring} algebraic structure, i.e. 2 operations, $*$ and $+$ with usual properties;
  \item define a differentiation rule $D$;
  \item be able to define arbitrary univariate polynomial functions combining the operators $*$, $+$, $D$;
  \item be able to lazily evaluate the function.
  \item \alert{Definition} happens \alert{at compile time};
  \item \alert{Evaluation} happens \alert{either at run time or at compile time}, depending on the argument.
  \end{itemize}
  We choose a \emph{polynomial-like} notation for the function, i.e.
  $f(x)=x(x-x^2+\frac{\partial(x^3+x)}{\partial x})$
  translates to
  \verb1auto f_x=x*(x-x*x+D(x*x*x+x));1.

  NOTE: \alert{using C++14 standard}.
\end{frame}

\begin{frame}[fragile]\frametitle{Example I: ET/AD}
  We define a placeholder, i.e. a type representing our independent variable $x$.
\begin{Cpplisting}[: define placeholder]{}
struct p{
  constexpr p(){};

  template<typename T>
  constexpr T operator() (T t_) const {
    return t_;
  }
};

constexpr p x=p();
\end{Cpplisting}
Global variable x=p(): to lighten the notation.
\end{frame}


\begin{frame}[fragile]\frametitle{Example I: ET/AD}
  define the plus expression and overload the $+$ operator
\begin{Cpplisting}[: sum expression]{}
template <typename T1, typename T2>
struct expr_plus{
  template<typename T>
  constexpr T operator() (T t_) const{
    return T1()(t_)+T2()(t_);
  }
};

template<typename T1, typename T2>
constexpr expr_plus<T1, T2>
operator + (T1 arg1, T2 arg2){
  return expr_plus<T1, T2 >();}
\end{Cpplisting}
NOTE: in real code use protections and namespaces
\end{frame}

\begin{frame}[fragile]\frametitle{Example I: ET/AD}
  define the times expression and overload the $*$ operator
\begin{Cpplisting}[: times expression]{}
template <typename T1, typename T2>
struct expr_times{
  template<typename T>
  constexpr T operator() (T t_) const{
    return T1()(t_)*T2()(t_);
  }
};

template<typename T1, typename T2>
constexpr expr_times<T1, T2>
operator * (T1 arg1, T2 arg2){
  return expr_times<T1, T2 >();}
\end{Cpplisting}
\end{frame}

\begin{frame}[fragile]\frametitle{Example I: ET/AD}
  main function:
  \begin{Cpplisting}[: main]{}
#include "p.hpp"
#include "expr_times.hpp"
#include "expr_plus.hpp"
int main(){
  constexpr auto expr = x*x*x; // define an expression
  std::cout<< expr(3)<<"\n"; // evaluate it in x=3
}
  \end{Cpplisting}
  $\approx20$ lines of code. How would that look like using template metaprogramming?

  That was easy.
\end{frame}

\begin{frame}[fragile]\frametitle{Example I: ET/AD}
  Now we want to compute the derivative: we use the \alert{chain rule}
  $$
  \frac{\partial f(g(x))}{\partial x}=\frac{\partial f(g)}{\partial g}\frac{\partial g(x)}{\partial x}
  $$
  i.e. we multiply the derivatives of each expression.

  Let's define the derivative of each expression, starting from $\frac{\partial x}{\partial x}=1$
  \begin{Cpplisting}[: x derivative]{}
template <typename T1>
struct expr_derivative;

template<>
struct expr_derivative<p>{

  template <typename T>
  constexpr T operator() (T t_) const{
    return (T) 1;
  }
};
  \end{Cpplisting}
\end{frame}

\begin{frame}[fragile]\frametitle{Example I: ET/AD}
  Derivative of the sum: $\frac{\partial (a+b)}{\partial x}=\frac{\partial (a)}{\partial x}+\frac{\partial (b)}{\partial x}$
  \begin{Cpplisting}[: sum derivative]{}
template <typename T1, typename T2>
struct expr_derivative<expr_plus<T1, T2> >{
    using value_t = decltype(D(T1())+ D(T2()));

    template <typename T>
    constexpr T operator() (T t_) const{
        return value_t()(t_);
    }
};
  \end{Cpplisting}
\end{frame}

\begin{frame}[fragile]\frametitle{Example I: ET/AD}
  Derivative of the product: $\frac{\partial (a*b)}{\partial x}=\frac{\partial (a)}{\partial x}*b+a*\frac{\partial (b)}{\partial x}$
  \begin{Cpplisting}[: product derivative]{}

template <typename T1, typename T2>
struct expr_derivative<expr_times<T1, T2> >{

    using value_t = decltype(T1() * D(T2())
                             +
                             D(T1()) * T2());

    template <typename T>
    constexpr T operator()(T t_)  const{
        return value_t()(t_);
    }
};
  \end{Cpplisting}
\end{frame}


\begin{frame}[fragile]\frametitle{Example I: ET/AD}
  Derivative of the derivative: $\frac{\partial^2 a}{\partial x} = \frac{\partial (\frac{\partial a}{\partial x})}{\partial x}$
  \begin{Cpplisting}[: second derivative]{}
template <typename T1>
struct expr_derivative<expr_derivative<T1>>{

    using value_t = decltype(D(D(T1())));

    template <typename T>
    constexpr T operator() (T t_) const{
        return value_t()(t_);
    }
};
  \end{Cpplisting}

  We also define the operator \verb1D1, returning the derivative expression
  \begin{Cpplisting}[: D operator]{}
template <typename T1>
constexpr expr_derivative<T1>
D (T1 arg1){
  return expr_derivative<T1>();}
  \end{Cpplisting}
\end{frame}

\begin{frame}[fragile]\frametitle{Example I: ET/AD}
  Now we can extend our main function with the derivative computation:
  \begin{Cpplisting}[: main]{}
#include "p.hpp"
#include "expr_times.hpp"
#include "expr_plus.hpp"
#include "expr_derivative.hpp"
int main(){
  constexpr auto expr = x*x*x; // define an expression
  std::cout<< expr(3)<<"\n"; // evaluate it in x=3
  std::cout<< D(expr)(2)<<"\n"; // evaluate it in x=2
  std::cout<< D(D(expr))(5)<<"\n"; // evaluate it in x=5
}
  \end{Cpplisting}
\end{frame}

\begin{frame}[fragile]\frametitle{Exercice I}
  Exercice: count (at compile time) the number of sums and multiplications of an expression,
  in order for the following to work:

  \begin{Cpplisting}[: Exercice I]{}
std::cout<< "sums: "<< D(expr).sum_ops()<<"\n";
std::cout<< "multiplications: "<< D(expr).mult_ops()<<"\n";
  \end{Cpplisting}
\end{frame}

\begin{frame}[fragile]\frametitle{Example II: Storage}
  Goal:
  \begin{itemize}
  \item Implement a multidimensional array indexing
  \item with arbitrary dimensionality;
  \item with strides computation happening at compile time when all dimension are compile-time constants;
  \item with an index function, returning a global storage index given the dimensions;
  \end{itemize}
  We considers 2 main objects:
  \begin{itemize}
    \item a \verb1storage_info1, containing the dimensionality, the strides, and the index member function;
    \item a \verb1storage1, trivial class containing the pointer to the data (\alert{not considered here});
  \end{itemize}

\end{frame}

\begin{frame}[fragile]\frametitle{Example II: Storage}
  This example is written using the C++14 standard, and in particular shows a practical use case for
  variadic templates and \verb1std::integer_sequence1. It relies on recursion (this should be familiar).

  \alert{Reminder:}
  \begin{Cpplisting}[: variadic templates]{}
  template<typename ... Ints>
  constexpr int index(Ints ... idx){
  \end{Cpplisting}
  allows specifying etherogeneous lists of types

  \begin{Cpplisting}[: integer sequence]{}
  template<int ... Indices, typename ... Ints>
  constexpr int idx_impl(std::integer_sequence< Indices ... > , Ints ... dims_) const;
  \end{Cpplisting}
  to assign an integer index to each of the elements in the \verb1dims_1 parameter pack ("zipping")
\end{frame}


\begin{frame}[fragile]\frametitle{Example II: Storage Info}
  \begin{Cpplisting}[: storage info]{}
template <int Size>
struct storage_info{
  template <typename ... Dims>
  constexpr storage_info(Dims ... dims_) : m_dims{dims_ ...}{}

  static constexpr int size(){return Size;}

  template<int Coord>
  constexpr int dim() const {return m_dims[Coord];}

  template<typename ... Ints>
  constexpr int index(Ints ... idx_) const{
    return idx_impl(std::make_integer_sequence<int, sizeof...(Ints)>(), idx_ ... );
  }

  template<int ... Indices, typename ... Dims>
  constexpr int idx_impl(std::integer_sequence< Indices ... > , Dims ... dims_) const;
private:
  int m_dims[Size];
};
  \end{Cpplisting}

\end{frame}


\begin{frame}[fragile]\frametitle{Example II: Strides Computation}
  Recursive constexpr member function to compute the strides
  \begin{Cpplisting}[: strides computation ]{}
template<int First, int ... Indices, typename First_Int, typename ... Ints>
constexpr int idx_impl(std::integer_sequence< int,  First, Indices ... > , First_Int first_, Ints ... indices) const {
  return (sizeof ... (Indices) > 0)
  ? first_ + m_dims[First] * idx_impl(std::integer_sequence<int, Indices ...>(), indices ...)
  : first_;
}
  \end{Cpplisting}
\end{frame}

\begin{frame}[fragile]\frametitle{Example II: Main}
  \begin{Cpplisting}[: main ]{}
#include "storage.hpp"
#include <iostream>

int main(){
    constexpr storage_info<5> c_{2,3,4,5,6};
    static_assert(c_.index(1,0,0,0,0)==1, "error");
    static_assert(c_.index(0,1,0,0,0)==2, "error");
    static_assert(c_.index(0,0,1,0,0)==3*2, "error");
    static_assert(c_.index(0,0,0,1,0)==4*3*2, "error");
    static_assert(c_.index(0,0,0,0,1)==5*4*3*2, "error");
  \end{Cpplisting}
\end{frame}

\begin{frame}[fragile]\frametitle{Exercice II}
  Abstract the memory layout of the storage, specifying it with the type \verb~layout_map<0,1,3,2,...>~,
  so that the following can work

  \begin{Cpplisting}[: Exercice II]{}
constexpr storage_info<layout_map<4,3,2,1,0> > c_{2,3,4,5,6};

static_assert(c_.index(1,0,0,0,0)==6*5*4*3, "error");
static_assert(c_.index(0,1,0,0,0)==6*5*4, "error");
static_assert(c_.index(0,0,1,0,0)==6*5, "error");
static_assert(c_.index(0,0,0,1,0)==6, "error");
static_assert(c_.index(0,0,0,0,1)==1, "error");
  \end{Cpplisting}
  To do so follow the instructions in the \verb1storage_info.hpp1 source file.
\end{frame}


% CHAPTER SLIDE
% \cscschapter{Chapter Title}

%% % Block style example
%% \begin{frame}{Quick Styles}
%% \begin{columns}
%%     \begin{column}{0.12\paperwidth}
%%     \begin{black0block}{Abc}
%%     \end{black0block}
%%     \begin{black1block}{Abc}
%%     \end{black1block}
%%     \begin{black2block}{Abc}
%%     \end{black2block}
%%     \end{column}

%%     \begin{column}{0.12\paperwidth}
%%     \begin{green0block}{Abc}
%%     \end{green0block}
%%     \begin{green1block}{Abc}
%%     \end{green1block}
%%     \begin{green2block}{Abc}
%%     \end{green2block}
%%     \end{column}

%%     \begin{column}{0.12\paperwidth}
%%     \begin{blue0block}{Abc}
%%     \end{blue0block}
%%     \begin{blue1block}{Abc}
%%     \end{blue1block}
%%     \begin{blue2block}{Abc}
%%     \end{blue2block}
%%     \end{column}

%%     \begin{column}{0.12\paperwidth}
%%     \begin{brown0block}{Abc}
%%     \end{brown0block}
%%     \begin{brown1block}{Abc}
%%     \end{brown1block}
%%     \begin{brown2block}{Abc}
%%     \end{brown2block}
%%     \end{column}

%%     \begin{column}{0.12\paperwidth}
%%     \begin{purple0block}{Abc}
%%     \end{purple0block}
%%     \begin{purple1block}{Abc}
%%     \end{purple1block}
%%     \begin{purple2block}{Abc}
%%     \end{purple2block}
%%     \end{column}

%%     \begin{column}{0.12\paperwidth}
%%     \begin{yellow0block}{Abc}
%%     \end{yellow0block}
%%     \begin{yellow1block}{Abc}
%%     \end{yellow1block}
%%     \begin{yellow2block}{Abc}
%%     \end{yellow2block}
%%     \end{column}

%%     \begin{column}{0.12\paperwidth}
%%     \begin{red0block}{Abc}
%%     \end{red0block}
%%     \begin{red1block}{Abc}
%%     \end{red1block}
%%     \begin{red2block}{Abc}
%%     \end{red2block}
%%     \end{column}

%% \end{columns}
%% \end{frame}

%% % Example
%% \begin{frame}{Formula}
%%     My formula is $E=m\times c^2$
%%     \begin{blue2block}{Are examples in a blue block?}
%%     \begin{itemize}
%%         \item My example
%%         \begin{itemize}
%%             \item for a formula
%%             \begin{itemize}
%%                 \item rocks!
%%             \end{itemize}
%%         \end{itemize}
%%     \end{itemize}
%%     \end{blue2block}

%%     \begin{red2block}{Alert an error in the formula!}
%%     \begin{enumerate}
%%         \item First enumerated item
%%         \begin{enumerate}
%%             \item First first enumerated item
%%             \begin{enumerate}
%%                 \item First first first enumerated item
%%             \end{enumerate}
%%         \end{enumerate}
%%         \item just joking...
%%     \end{enumerate}
%%     \end{red2block}
%% \end{frame}

%% \begin{frame}[fragile]{Code listing}
%% \begin{Cpplisting}[: My code title with some openmp]{}
%% int main (int argc, char *argv[]) {
%%     int nthreads, tid;
%%     #pragma omp parallel private(nthreads, tid) {
%%         tid = omp_get_thread_num();
%%         printf("Hello World from thread = %d\n", tid);
%%         if (tid == 0) {
%%             nthreads = omp_get_num_threads();
%%             printf("Number of threads = %d\n", nthreads);
%%         }
%%     }  /* All threads join master thread and disband */
%%     return 0;
%% }
%% \end{Cpplisting}
%% \begin{Fortranlisting}[: My Fortran code here]{}
%% program variable
%% real :: x ! This defines a variable called x
%% x = 23.6*log(4.2)/(3.0+2.1) ! This gives it a value
%% print *, 'The value of x is ', x
%% end program
%% \end{Fortranlisting}
%% \footnotesize
%% Here some cpp code in the text: \lstinlineCpp{int main(int args, char *argv[])}\\
%% Here some Fortran code in the text: \lstinlineFortran{INTEGER :: Y(1,5)}
%% \end{frame}

% THANK YOU SLIDE
\cscsthankyou{Thank you for your attention.}

\end{document}
